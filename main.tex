\documentclass{beamer}
\usetheme{NIU}

% suppress navigation bar
\beamertemplatenavigationsymbolsempty

% Title
\title[Short Title]
{Long Title}

% Sub Title
%\subtitle{a}

% Author
\author[F. Author]
{\texorpdfstring{\underline{First~Author}}{First~Author}\inst{\dag} \and Second~Author\inst{\dag} \and Third~Author\inst{\ddag}}
\hypersetup{pdfauthor={Ramanpreet Singh}}

% - Give the names in the same order as the appear in the paper.
% - Use \and to separate authors name
% - Use the \inst{?} command only if the authors have different affiliation.

\institute[NIU]% (optional)
{
 \inst{\dag}%
   Northern Illinois University, USA
   \and
   \inst{\ddag}%
   Lorem Ipsum
}
% - Use the \inst command only if there are several affiliations.
% - Keep it simple, no one is interested in your street address.

\date{March 17, 2018}
\subject{} % only for pdf info

%\AtBeginSubsection[]
%{
%  \begin{frame}<beamer>{Outline}
%    \tableofcontents[currentsection,currentsubsection]
%  \end{frame}
%}


% Title Graphics add befor titlepage
\titlegraphic{\includegraphics[height=3cm]{NIU_logo.eps}}

\begin{document}
\frame{\titlepage}


% add logo after title page, so it does show on title page
\logo{\includegraphics[trim=1cm 2.5cm 1cm 0,clip,width=2cm]{NIU_logo.eps}}

% Let's get started

\begin{frame}<beamer>{Outline}
    \tableofcontents
\end{frame}

\section{First Main Section}

\subsection{First Subsection}

\begin{frame}{First Slide Title}{Optional Subtitle}
  \begin{itemize}
  \item {
    My first point.
  }
  \item {
    My second point.
  }
  \end{itemize}
\end{frame}

\subsection{Second Subsection}

% You can reveal the parts of a slide one at a time
% with the \pause command:
\begin{frame}{Second Slide Title}
  \begin{itemize}
  \item {
    First item.
    \pause % The slide will pause after showing the first item
  }
  \item {   
    Second item.
  }
  % You can also specify when the content should appear
  % by using <n->:
  \item<3-> {
    Third item.
  }
  \item<4-> {
    Fourth item.
  }
  % or you can use the \uncover command to reveal general
  % content (not just \items):
  \item<5-> {
    Fifth item. \uncover<6->{Extra text in the fifth item.}
  }
  \end{itemize}
\end{frame}

\section{Second Main Section}

\subsection{Another Subsection}

\begin{frame}{Blocks}
\begin{block}{Block Title}
You can also highlight sections of your presentation in a block, with it's own title
\end{block}
\begin{theorem}
There are separate environments for theorems, examples, definitions and proofs.
\end{theorem}
\begin{example}
Here is an example of an example block.
\end{example}
\begin{alertblock}{Alert}
Here is an example of an alert block.
\end{alertblock}
\end{frame}

\end{document}